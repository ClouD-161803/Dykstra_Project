\section{The Polyhedral Case}
\label{sec:polyhedral_case}
Several simplifications to Dykstra's algorithm are available for polyhedral sets $\mathcal{H}_i$ (as defined in~\eqref{eq:polyhedron}). The remainder of this paper will narrow the analysis to focus on these types of convex subsets.

For polyhedral sets~\eqref{eq:sets_i}, the projection step~\eqref{eq:dykstra:proj} can be simplified to:
\begin{align}\label{eq:dykstra:proj:poly}
x_{m+1}=
\begin{cases}
x_{m}+e_{m-n} & \text{if } x_{m}+e_{m-n}\in\mathcal{H}_{[m]}\\
x_{m}+e_{m-n} - \left((x_{m}+e_{m-n})^T f_{[m]} - c_{[m]}\right) f_{[m]} & \text{if } x_{m}+e_{m-n}\not\in\mathcal{H}_{[m]}
\end{cases},
\end{align}
and the update for the auxiliary vector~\eqref{eq:dykstra:error} to:
\begin{align}\label{eq:dykstra:error:poly}
e_m=
\begin{cases}
0 & \text{if } x_{m}+e_{m-n}\in\mathcal{H}_{[m]},\\
\left((x_{m}+e_{m-n})^T f_{[m]} - c_{[m]}\right) f_{[m]} & \text{if } x_{m}+e_{m-n}\not\in\mathcal{H}_{[m]},
\end{cases}.
\end{align}
The auxiliary vector $e_m$ is either $0$ or parallel to $f_{[m]}$, so that it can be represented as $e_m = k_m f_{[m]}$ with $k_m=\text{dist}_{\mathcal{H}_{[m]}}(x_{m-1}+e_{m-n})$, further simplifying~\eqref{eq:dykstra:error:poly} to:
\begin{align}\label{eq:km}
k_m = k_{m-n} + x_m^T f_{[m]} - c_{[m]}.
\end{align}

The convergence of Dykstra's iterates to the Eucledian projection has been analysed in~\cite{DYKSTRAPOLY,DYKSTRAPOLY2,DYKSTRAPERKINS} for polyhedral sets. The convergence proof in~\cite{DYKSTRAPOLY} is based on partitioning the sets into inactive ($x^\star\not\in H_i$) and active sets ($x^\star\in H_i$), i.e.:\
\begin{align}
&A=\set{i\in\lbrace 0,\dots,n-1\rbrace}{x_\infty\in H_i},
&B=\lbrace 0,\dots,n-1\rbrace\backslash A,
\end{align}
where $x_\infty=\lim_{m\rightarrow\infty}x_m$. It can be shown that there exists a number $N_1$ such that whenever:
\begin{align}
[m]\in B,\quad m\geq N_1\quad\Rightarrow\quad x_m=x_{m-1},\quad e_m=0,
\end{align}
i.e. the half-spaces that become ``inactive'' remain inactive. Furthermore, there exists $N_2\geq N_1$ such that whenever $n\geq N_2$, it holds that
\begin{align}\label{eq:dykstra:error:poly2}
\twonorm{x_{m+n}-x_\infty}\leq\alpha_{[m]}\twonorm{x_m-x_\infty},
\end{align}
where $0\leq\alpha_{[m]}<1$ are numbers related to angles between half-spaces. The number $N_2$ describes the iteration from which on the algorithm has determined the inactive half-spaces. Finally, it is shown that the iterates of the algorithm satisfy the following inequality:
\begin{theorem}[Deutsch and Hundal~\cite{DYKSTRAPOLY}]
There exist constants $0\leq c < 1$ and $\rho > 0$ such that
\begin{align*}
\anynorm{x_m -x_\infty} \leq \rho c^m.
\end{align*}
\end{theorem}
The factor $c$ can be estimated from the smallest $\alpha_{[m]}$, which is characterized by the angle between certain subspaces (subspaces formed by the ``active'' halfspaces). The factor $\alpha_{[m]}$ can be upper-bounded by considering the ``worst'' angles in the polyhedron. The constant $\rho$, however, depends on an unknown iteration number $N_3\geq N_2$ and on the starting point $x^\circ$, and can therefore not be computed in advance~\cite{DYKSTRAPERKINS,XUPOLY}. In the case of stalling, the variable $\rho$ can become arbitrarily large, making the application of Dykstra's method difficult in practice. The authors of~\cite{DYKSTRAPERKINS} proposed a combined Dysktra-conjugate-gradient method that allows for computing an upper bound on $\anynorm{x_m -x_\infty}$. The authors of~\cite{XUPOLY} proposed an alternative algorithm called \emph{successive approximate algorithm}, which promises fast convergence, conditioned on knowing a point $x\in\mathcal{H}$ in advance.

%\begin{remark}[Random thoughts]
%Dykstra's method could be interpreted as an autonomous non-linear discrete-time system with initial condition $x^\circ$. For $m\geq N_2$, I believe that the discrete-time system becomes \emph{linear} and could be reformulated as $x_{m+n} = G x_m + b$.
%Since by the Boyle-Dykstra theorem $x_{m+n}$ necessarily converges to the projection for $n\rightarrow\infty$, the matrix $G$ must be Schur stable, so that the projection can be obtained from $x^\star = \inv{\left(I-G\right)} b$. The question is whether $N_2$ can be detected (may have been answered in the literature already).
%\end{remark}

% Figure demonstrating the stalling problem
\begin{figure}[h]
    \centering
    \begin{subfigure}[t]{0.49\textwidth}
        \centering
        \includegraphics[width=1\textwidth]{Latex/Current Version/Figures/StallingRegionsHand.png}
        \caption{Line-box example with different regions that yield different convergence properties.}
        \label{fig:region}
    \end{subfigure}
    \hfill
    \begin{subfigure}[t]{0.49\textwidth}
        \centering
        \includegraphics[width=1\textwidth]{Latex/Current Version/Figures/DifferentSequences.png}
        \caption{Stalling for the line-box example when $x_0$ is in the red region.}
        \label{fig:stalling}
    \end{subfigure}
    \caption{A demonstration of the stalling problem for a box and a line. Note how MAP applied to the same constraint sets would not result in any stalling: MAP follows the green line, and subsequently converges via the blue line path. Figure taken from~\cite{DYKSTRASTALLING}.}
    \label{fig:baushkeStall}
\end{figure}

\subsection{Stalling}

In~\cite{DYKSTRASTALLING}, the behaviour of Dykstra's method is analysed for two sets. The authors give conditions on Dykstra's algorithm for (i) finite convergence, (ii) infinite convergence, and (iii) stalling followed by infinite convergence. A specific example is given for the case that the set is provided by the intersection of a line with a unit box in $\R^2$ ($\mathcal{H}$ is a polyhedron). It can be shown that cases (i)--(iii) depend on the starting point $x_0$, and one can determine the 3 regions shown in Figure~\ref{fig:region} that yield different convergence behaviour. Convergence case (i) is obtained when starting in the green region, case (ii) when starting in the blue region, and case (iii) when starting in the red region.

To understand the stalling effect, consider Figure~\ref{fig:stalling}, which shows the first iterations of Dykstra's algorithm with starting point in the red region. Note that the outcome of Dykstra's algorithm depends on the order of the sets $\mathcal{H}_i,\dots,\mathcal{H}_n$. In Figure~\ref{fig:stalling}, the algorithm starts by projecting onto the box and then onto the line. It can be seen that for the first 6 iterations\footnote{By one iteration we mean one cycle of $n$ projections here.}, Dykstra's algorithm returns the top left corner of the box (``stalling''). The authors also determine the exact number of iterations required to break free from the red region, and show that if the starting point is arbitrarily far to the left, the algorithm will need an arbitrarily large iteration number to break free from the red region.